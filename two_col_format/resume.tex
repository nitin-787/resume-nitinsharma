\documentclass[]{deedy-resume-openfont}
\usepackage{fancyhdr}
 
\pagestyle{fancy}
\fancyhf{}
 
\begin{document}

\namesection{Nitin}{Sharma}{ \urlstyle{same}\href{mailto:nitinsharma.devx@gmail.com}{nitinsharma.devx@gmail.com} | \href{https://github.com/nitin-787}{Github nitin-787} | 
\href{https://www.linkedin.com/in/nitin787/}{LinkedIn nitin787}
}

\begin{minipage}[t]{0.33\textwidth} 

\section{Education} 

\runsubsection{Rayat Bahra University}
\descript{B.TECH in Computer Science}
\location{2023 - 2026}
\sectionsep

\runsubsection{Lovely University} \\
\descript{Diploma in Computer Science}
\location{2020 - 2023 | 7.84/10 CGPA}
\sectionsep

\runsubsection{Sacred Touch} \\
\descript{Std X | 2020 | A+ Grade}
\sectionsep

\section{Links} 
Github:// \href{https://github.com/nitin-787}{\bf nitin-787} \\
LinkedIn://  \href{https://www.linkedin.com/in/nitin787/}{\bf nitin787} \\
LeetCode://  \href{https://leetcode.com/u/random_potato/}{\bf randompotato} \\
Codeforces://  \href{https://codeforces.com/profile/random_potato}{\bf  randompotato}\\

\section{Skills}
\runsubsection{Programming} \\
\textbullet{} Java \textbullet{}   C \textbullet{} C++ \textbullet{} Python \\ \textbullet{} JavaScript \textbullet{} Dart  \\\sectionsep

\runsubsection{Tech Stack} \\
\textbullet{} Android Dev \textbullet{} Web Dev \\ \textbullet{} Flutter \textbullet{} Firebase \textbullet{} Blender \\
\sectionsep

\runsubsection{Tools} \\
\textbullet{} Git/Github \textbullet{} Docker \\ \textbullet{} Ubuntu \textbullet{} CI/CD \\ \textbullet{} Custom ROMs/Rooting \\
\sectionsep

\runsubsection{Familiar} \\
\textbullet{} React JS \textbullet{} PHP \textbullet{} MySql \\ \textbullet{} Kubernetes
\sectionsep

% coding plateform rating
\runsubsection{Competitive Programming} \\
\textbullet{} Codeforces: 1134 \\
\textbullet{} LeetCode: 1464 \\
\textbullet{} CodeChef: 1173 \\
\sectionsep

\subsection{Languages} \\
\textbullet{} English \textbullet{} Hindi \textbullet{} Punjabi
\sectionsep

%%%%%%%%%%%%%%%%%%%%%%%%%%%%%%%%%%%%%%
%
%     COLUMN TWO
%
%%%%%%%%%%%%%%%%%%%%%%%%%%%%%%%%%%%%%%

\end{minipage} 
\hfill
\begin{minipage}[t]{0.66\textwidth} 

\section{Experience}
\runsubsection{Google Summer of Code'22}
\descript{| Contributor | (Unofficial)}
\location{Jun 12, 2022 – Oct 19, 2022 | \href{https://github.com/MarkisDev/its_urgent} {Github}}
\begin{itemize}[leftmargin=*]
    \item Led integration tasks in \textbf{GSoC'22 @CCExtractor,} enhancing the CI/CD pipeline to optimize deployment efficiency and minimize integration issues.
    \vspace{-2.3mm} % Adds space between bullet points
    \item Managed the \textbf{CI/CD pipeline}, reducing build failures by \textbf{30\%} and streamlining software deployments.
    \vspace{-2.3mm}
    \item \textbf{Fixed bugs} that occurred during the development of the Project.
    \vspace{-2.3mm}
    \item Enhanced project functionality by \textbf{contributing 16 well-documented commits,} focusing on modularity and maintainability.
\end{itemize}

\sectionsep

\runsubsection{Hacktoberfest}
\descript{| Project Maintainer}
\location{Oct 2022 – Oct 2022 | \href{https://github.com/nitin-787/uni-notes/graphs/contributors} {Github}} 
\begin{itemize}[leftmargin=*]
    \item \textbf{Mentored 15 first-time contributors} during Hacktoberfest, providing guidance on submitting high-quality pull requests.
    \vspace{-2.3mm} % Adds space between bullet points
    \item \textbf{Led the open-source development process,} helping contributors understand best practices and contributing to \textbf{5+ successful PRs.}
    \vspace{-2.3mm}
    \item \textbf{Assisted in merging 20+ pull requests,} fostering collaboration and contributing to the project's growth.
\end{itemize}
\sectionsep

\runsubsection{Geeks for Geeks}
\descript{| Technical Content Writer }
\location{Apr 2023 - sept 2023 | Link: \href{https://auth.geeksforgeeks.org/user/nitinsharma29/articles}{\textbf{Articles}}} 
\vspace{\topsep}
\begin{itemize}[leftmargin=*]
    \item Authored \textbf{3} technical articles on Docker and AWS, which received over \textbf{7500+} views and contributed to reader's understanding of cloud computing.
\end{itemize}

\sectionsep

\section{Projects}
\runsubsection{Uni-Share}
\location{| Flutter | Firebase | \href{https://github.com/nitin-787/uni-notes/graphs/contributors} {Github}}
% \vspace{\topsep} 
\begin{itemize}[leftmargin=*]
    \item \textbf{Developed the Uni-Share app prototype,} aiming to support \textbf {1000+ users} for sharing academic resources like notes, assignments.
    \vspace{-2.3mm} % Adds space between bullet points
    \item \textbf{Implemented 10+ core features,} including content sharing, real-time chat, and file uploads, enhancing student collaboration.
    \vspace{-2.3mm}
    \item \textbf{Integrated Firebase} enabling real-time data syncing, robust database management, and seamless backend services.
\end{itemize}
\sectionsep

\runsubsection{Ayumi GPT}
\location{| Flutter | Open AI |
\href{https://github.com/nitin-787/Ayumi-gpt}{Github}}
% \vspace{\topsep} 
\begin{itemize}[leftmargin=*]
    \item \textbf{Developed Ayumi GPT,} a mobile app powered by OpenAI’s GPT model, enabling real-time, interactive conversations.
    \vspace{-2.4mm} % Adds space between bullet points
    \item \textbf{Integrated Chat GPT API,} enhancing user interactions with responsive AI-driven replies.
    \vspace{-2.4mm}
    \item \textbf{Implemented key features,} like personalized chat histories and suggestions, increasing user engagement.
\end{itemize}
\sectionsep

\end{minipage} 
\end{document}  \documentclass[]{article}
